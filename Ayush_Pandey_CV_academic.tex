
\documentclass[10pt,a4paper,sans]{moderncv}        % possible options include font size ('10pt', '11pt' and '12pt'), paper size ('a4paper', 'letterpaper', 'a5paper', 'legalpaper', 'executivepaper' and 'landscape') and font family ('sans' and 'roman')

% Custom dictionary for spell check
\newcommand{\ignorewords}[1]{}
\ignorewords{moderncv oldstyle lualatex Docteur Système Distribué Accroche CRDTs CRDT Riak lockdowns}


% moderncv themes
\moderncvstyle{classic}                             % style options are 'casual' (default), 'classic', 'oldstyle' and 'banking'
\moderncvcolor{blue}                               % color options 'blue' (default), 'orange', 'green', 'red', 'purple', 'grey' and 'black'

% character encoding
\usepackage[utf8]{inputenc}                       % if you are not using xelatex ou lualatex, replace by the encoding you are using
% For the three columns text
\usepackage{multicol}

\newboolean{long}
\setboolean{long}{false}

% adjust the page margins
\usepackage[scale=0.75, top=35pt, bottom=35pt, left=55pt, right=55pt]{geometry}
%\setlength{\hintscolumnwidth}{3cm}                % if you want to change the width of the column with the dates
%\setlength{\makecvtitlenamewidth}{10cm}           % for the 'classic' style, if you want to force the width allocated to your name and avoid line breaks. be careful though, the length is normally calculated to avoid any overlap with your personal info; use this at your own typographical risks...

% personal data
\renewcommand*{\namefont}{\fontsize{24}{20}\mdseries\upshape}
\renewcommand*{\emailsymbol}{}
\name{Ayush}{\mbox{Pandey}}


\title{Distributed Systems Researcher}
%\title{Docteur en Système Distribué}
\address{75013 Paris}{}% optional, remove / comment the line if not wanted; the "postcode city" and and "country" arguments can be omitted or provided empty
% \phone[mobile]{07 81 49 75 76}                   % optional, remove / comment the line if not wanted
\social[linkedin][www.linkedin.com/in/ayushpandey8439/]{www.linkedin.com/in/ayushpandey8439/}
\email{ayushpandey8439@gmail.com}                               % optional, remove / comment the line if not wanted
% \extrainfo{French}
% \photo[64pt][0pt]{Saalik}                       % optional, remove / comment the line if not wanted; '64pt' is the height the picture must be resized to, 0.4pt is the thickness of the frame around it (put it to 0pt for no frame) and 'picture' is the name of the picture file
%\quote{\mbox{PhD} }                           % optional, remove / comment the line if not wanted

%----------------------------------------------------------------------------------
%            content
%----------------------------------------------------------------------------------
\begin{document}
%-----       resume       ---------------------------------------------------------
\makecvtitle
%% Accroche

\vspace{-1cm}

\section{Research Interests}

\cventry{}{Distributed Systems, Consistency protocols, CRDTs, Graph Databases}{}{}{}{}
My primary interest lies at the intersection of distributed systems and connected data. I am particularly interested in the design, implementation, and verification of replication and consistency protocols for distributed systems. During my PhD, I have worked on multi-granularity locking in dynamic hierarchies and developed CALock, a new locking protocol for parallel threads over hierarchical data. I am collaborating with researchers from NOVA LINCS and RPTU Kaiserslautern to specify the correctness of transactional database backends. Alongside, with Telecom Sud-Paris on designing and developing a schema-first, local-first graph system to facilitate collaborative knowledge graph editing.


\section{Publications}
\cventry{2026}{Safety of Database Backend Stores}{ICDT}{(Preprint)}{}{}
\cventry{2025}{Towards Local-First Distributed Property Graphs}{PaPoC}{}{}{}
\cventry{2025}{CALock: Multi-granularity locking in dynamic hierarchies}{IPDPS}{}{}{}
\cventry{2024}{Diversifying locks for effective synchronization in dynamic graphs}{EuroSys DW}{}{}{}
\cventry{2023}{Verrouillage multi-granularité dans les graphes orientés}{ComPas}{}{}{}
% \cventry{2022}{Persisting the AntidoteDB Cache}{Inria}{Tech Report}{}{}



% \section{Skills}
% \closesection{}


% \begin{multicols}{3}
%     {\large \color{color1} General Skills}\\
%     \textbf{Distributed systems}\\
%     \textbf{Concurrent programming}\\
%     \textbf{Consistency protocols}\\
%     \textbf{Storage systems}\\
%     \textbf{CRDTs}\\
%     \textbf{Graph databases}\\
%     \textbf{Non-relational data models}\\
    

% \vfill\null\columnbreak


% {\large \color{color1}  Tools} \\
%     Neo4j\\
%     Gremlin\\
%     NoSQL\\
%     PostgreSQL\\
%     Redis\\
%     Docker\\ 
%     Git\\ 
%     UNIX environnement\\ 

% \vfill\null\columnbreak

% {\large \color{color1}  Languages}\\
% \textbf{Java}\\ 
% \textbf{C++}\\
% \textbf{Erlang, Elixir}\\
% Python\\
% C\\
% Bash\\

% {\large \color{color1} Speaking}\\
%     Hindi -Native\\
%     English - Fluent\\
%     French - Conversational\\    

% \vfill\null\columnbreak

% \end{multicols}



% \vspace{-0.8cm}

\section{Experience}

\cventry{2022-2025}{PhD Researcher}{LIP6, Sorbonne Université (UPMC)}{Paris}{}{
    \begin{itemize}
        \item \textbf{Thesis}: \textbf{CALock: topological multi-granularity locking in hierarchies}
        \item \textbf{Designed, implemented and verified a new locking protocol for parallel threads over hierarchical data.} CAlock identifies an optimal grain size for locking in a hierarchy of data structures.
        \item Implemented and verified several thread synchronization protocols in java and C\texttt{++} as well graph database implementations in TinkerPop(gremlin) as proof of concept.
        Performed benchmarks for performance evaluation.
        \item Collaboration in projects involving \textbf{industry partners and research labs}. Ongoing collaboration with Telecom sud-Paris to design, implement and formally verify a distributed graph database. Second ongoing work within LIP6 on the specification and verification of a distributed key-value store. 
    \end{itemize} 
}

\cventry{2021-2022}{Research Intern}{LIP6, Sorbonne Université (UPMC)}{Paris}{}{
    \begin{itemize}
        \item \textbf{Implementation of a cache for a CRDT datastore}. Worked on designing and implementing an in memory cache and checkpoint store for AntidoteDB. 
        \item Undertook performance benchmarking in a distributed environment with Riak Bench. 
        \item Fixed several bugs in the general implementation of AntidoteDB and achieved a 40\% performance improvement in the read path. 
        \item \textbf{Tech Report:} \href{https://inria.hal.science/hal-03654003v2}{Persisting the AntidoteDB Cache}
    \end{itemize}
}

\cventry{2020-2021}{Full stack Web Engineer}{LernFair e.v.}{Remote, Germany}{}{
    \begin{itemize}
        \item Worked on the development of an online learning platform for school students to facilitate remote learning during Covid-19 lockdowns. 
        \item Implemented the frontend in ReactJS and the backend in NodeJS and GraphQL along with alpha and beta testing of features. 
        \item Implemented the service layer in a microservice architecture with content delivery through REST APIs. 
    \end{itemize}
}

\cventry{2020-2021}{Research Assistant}{German Center for AI (DFKI)}{Kaiserslautern, Germany}{}{
    \begin{itemize}
        \item Developed a custom tool for visual graph editing tool with automated graph structuring and enforcing P\&ID constraints on nodes and edges of the graph in AngularJS and D3.js.
        \item Writing test suites for performance. testing on peak loads and high data volumes with python.
    \end{itemize}
}


\cventry{2017-2019}{Software Engineer}{Newgen Software Technologies}{India}{}{
    \begin{itemize}
        \item Development of custom workflow, content management and business process automation solutions for major Banks, Health care providers and insurance companies.
        \item Developed custom dashboards and reports for telemetry and report generation.
        \item Delivered technical training and support to business clients in the asia pacific region.
    \end{itemize}
}

% \cventry{2015-2016}{Product Manager}{ookee (Shutdown)}{Paris}{}{
%     ookee was an Android smartphone startup building a product for senior users.
%     Product manager of services, from business development to vulgarization to designing the products. 
% }



\section{Education}
\cventry{2025}{PhD in Distributed systems}{LIP6, Sorbonne Université (UPMC)}{France}{}{}
\cventry{2022}{Master in CS, Software Engineering}{TU Kaiserslautern}{Germany}{}{}
\cventry{2017}{Bachelor of Technology}{APJ Abdul kalam technical university}{India}{}{}

% \section{Hobbies}
% \cvitem{}{Running, Reading, Cooking, Karaoke}


\end{document}

